\documentclass[english, kiv, sem, he, iso690alph, pdf, viewonly]{fasthesis}
\title{Lisp language subset interpret}
\author{Jan}{Hejdušek}
\supervisor{Ing. Kamil Ekštein, Ph.D.}
\assignment{sw2025-02.pdf}

\usepackage{csquotes}
\usepackage{pdfpages}

\nobastardtitle
\nocopyrightnotice

\newif\iffullbuild
\fullbuildtrue

\begin{document}
  \iffullbuild
    \frontpages[notm]
    \tableofcontents
  \fi

  \chapter{Introduction}
    \section{Overview}
      This paper is a documentation for a seminar work in course KIV/PC -- Programming in C language.
      I will cover the assignment itself, analysis of the problem, details of the implementation, 
      short guide for building and running the application and finally a conclusion covering how the work went.
    \section{Assignment}
      The assignment for this seminar work is a console application written in programming language C
      capable of interpretting a subset of Lisp programming language(furthermore just Lisp). The exact details and defined subset of Lisp

  \chapter{Analysis}
    \section{Lisp}
    \section{Grammar}
    \section{Problem Statement}
    The goal of this project is to implement a simple Lisp interpreter in C, supporting basic arithmetic, logical operations, variable management, and control flow. The interpreter parses Lisp-like expressions, builds an abstract syntax tree (AST), and evaluates expressions in a managed environment.

    \section{Showcase: Complex Mathematical Formula}

    To demonstrate \LaTeX's capabilities, here is an intentionally complicated mathematical formula:

    \begin{equation}
    \begin{split}
      F(x) = \int_{-\infty}^{\infty} \left[ \sum_{n=1}^{\infty} \frac{(-1)^n}{n^2} \left( \prod_{k=1}^{n} \sqrt[k]{\frac{x^k + \sin(kx)}{e^{x/k} + k^2}} \right) \right] e^{-x^2} \, dx \\
      + \lim_{m \to \infty} \left\{ \frac{1}{m} \sum_{j=1}^{m} \left[ \frac{\Gamma(jx) + \zeta(j)}{\sqrt{j! + x^j}} \right] \right\}
    \end{split}
    \end{equation}

    where $\Gamma$ is the gamma function and $\zeta$ is the Riemann zeta function.

  \chapter{Implementation}
    \section{Architecture Overview}
    The interpreter is divided into several modules:
    \begin{itemize}
      \item \textbf{Lexer}: Tokenizes input strings into meaningful tokens for parsing.
      \item \textbf{Preprocessor}: Removes comments and normalizes input (e.g., uppercase conversion).
      \item \textbf{Parser}: Converts token arrays into an AST according to the defined grammar.
      \item \textbf{AST Management}: Provides functions for creating, copying, and freeing AST nodes.
      \item \textbf{Environment}: Manages variable storage and lookup.
      \item \textbf{Operators}: Implements built-in functions and arithmetic/logical operations.
      \item \textbf{REPL}: Handles user input, manages multi-line expressions, and invokes the interpreter.
    \end{itemize}
      % You can reference this section using \ref{sec:overview} if you add a label:
      \label{sec:overview}

      % Example diagram using tikz
      \begin{figure}[h]
        \centering
        \begin{tikzpicture}
          \node (start) [draw, rectangle] {Start};
          \node (parse) [draw, rectangle, right=1cm of start] {Parse};
          \node (eval) [draw, rectangle, right=1cm of parse] {Evaluate};
          \draw[->] (start) -- (parse);
          \draw[->] (parse) -- (eval);
        \end{tikzpicture}
        \caption{Interpreter workflow}
        \label{fig:workflow}
      \end{figure} 
      \subsection{Used data structures}

    \section{Macros and Error handling}

    \section{Handling input}
      \subsection{Arguments}
      \subsection{Repl}
    
    \section{Preprocessor}

    \section{Lexer}

    \section{Parser}

    \section{Evaluation}
      \subsection{Environment}
      \subsection{Operators}

    \section{Memory management}

  \chapter{User guide}
  \chapter{conclusion}
  \iffullbuild
    \appendix{}
    \includepdf[pages=-]{sw2025-02.pdf}
  \fi
\end{document}