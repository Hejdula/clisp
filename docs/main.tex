\documentclass[english, kiv, sem, he, iso690alph, pdf, viewonly]{fasthesis}
\title{Lisp language subset interpret}
\author{Jan}{Hejdušek}
\supervisor{Ing. Kamil Ekštein, Ph.D.}
\assignment{sw2025-02.pdf}

\usepackage{csquotes}
\usepackage{pdfpages}

\nobastardtitle
\nocopyrightnotice

\newif\iffullbuild
\fullbuildtrue

\begin{document}
  \iffullbuild
    \frontpages[notm]
    \tableofcontents
  \fi

  \chapter{Introduction}
    \section{Overview}
      This paper is a documentation for a seminar work in course KIV/PC -- Programming in C language.
      I will cover the assignment itself, analysis of the problem, details of the implementation, 
      short guide for building and running the application and finally a conclusion covering how the work went.
    \section{Assignment}
      The assignment for this seminar work is to develop a console application in the C programming language
      that interprets a subset of the Lisp programming language (hereafter referred to only as Lisp).
      The complete assignment, in Czech, is provided at the end of this document.

  \chapter{Analysis}
    \section{Problem Statement}
      We need to process an
      % To interpret a Lisp source code, we need to go through several steps. 
    \section{Grammar}
      Lisp strictly follows a fully parenthesized prefix notation. Lisp expressions can recursively contain other Lisp expressions.
      Expressions in our Lisp subset consist only of three possible components:
      \begin{itemize}
        \item Constant
        \item Variable
        \item List
      \end{itemize}
    


  \chapter{Implementation}
    \section{Architecture Overview}
    The interpreter is divided into several modules:
    \begin{itemize}
      \item \textbf{Preprocessor}: Removes comments and normalizes input (e.g., uppercase conversion).
      \item \textbf{Lexer}: Tokenizes input strings into meaningful tokens for parsing.
      \item \textbf{Parser}: Converts token arrays into an AST according to the defined grammar.
      \item \textbf{AST Module}: Provides functions for creating, copying, and freeing AST nodes.
      \item \textbf{Environment}: Manages variable storage and lookup.
      \item \textbf{Operators}: Implements built-in functions and arithmetic/logical operations.
      \item \textbf{REPL}: Handles user input, manages multi-line expressions, and invokes the interpreter.
    \end{itemize}

      % Example diagram using tikz
      \begin{figure}[h]
        \centering
        \begin{tikzpicture}
          \node (begin) [draw, rectangle] {Start};
          \node (preproc) [draw, rectangle, right=1.5cm of begin] {Preprocessor};
          \node (lexer) [draw, rectangle, right=1.5cm of preproc] {Lexer};
          \node (parse) [draw, rectangle, right=1.5cm of lexer] {Parser};
          \node (eval) [draw, rectangle, right=1.5cm of parse] {Evaluate};
          \draw[->] (begin) -- (preproc);
          \draw[->] (preproc) -- (lexer);
          \draw[->] (lexer) -- (parse);
          \draw[->] (parse) -- (eval);
        \end{tikzpicture}
        \caption{Interpreter workflow}
        \label{fig:workflow}
      \end{figure}
      \subsection{Used data structures}

    \section{Macros and Error handling}

    \section{Handling input}
      \subsection{Arguments}
      \subsection{Repl}
    
    \section{Preprocessor}

    \section{Lexer}

    \section{Parser}

    \section{Evaluation}
      \subsection{Environment}
      \subsection{Operators}

    \section{Memory management}

  \chapter{User guide}
  \chapter{conclusion}
  \iffullbuild
    \appendix{}
    \chapter{}
    \label{assignment}
    \includepdf[pages=-]{sw2025-02.pdf}
  \fi
\end{document}